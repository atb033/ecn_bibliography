%-------------TO BE COMPLETED-----------------
%Path planning is the problem of finding a sequence of \textit{actions} to transform a system from an initial state to a goal state. The planning problem has been studied extensively in various fields like robotics, artificial intelligence, and control theory. In this chapter, we discuss the fundamentals of path planning.

%\section{Terminology}
%\subsubsection*{Configuration Space}
%The path planning problem can be precisely formulated by defining a world, $\mathcal{W}$, in which the robot $\mathcal{A}$ exists, an obstacle region, $\mathcal{O}\subset\mathcal{W}$, the \textit{configuration space}, $\mathcal{C}$, 
%First, we study the popular algorithms used for discrete space planning. Then we 

%A few concepts necessary to precisely define the path planning problem are now introduced. The world ($\mathcal{W}$) is a space in which the robot ($\mathcal{A}$) exists. 


%The configuration space of a robot is the set of all configurations that could be achieved by it. For instance, let us consider a 2D robot that operates on a plane. Its configuration space is the 3D space defined by the special Euclidean group SE(2). 

%-------------- TO BE COMPLETED-------------------
\section{Terminology}
Each distinct situation of a world is called a \textit{state}, $x$, and the set of all possible states is called a \textit{state space}, $X$. The state, $x$, can be transformed to $x'$, by applying an \textit{action}, $u$, as specified by a \textit{state transition function}, $f$, such that:
\begin{align}
	x' = f(x,u)
\end{align}
\section{Discrete Space Planning Algorithms}

\section{Planning with Differential Constraints}
