Quadrotor helicopters has been a topic of extensive research work during the last decade. Their agility and robustness, attributed to their mechanical simplicity, makes them the preferred choice for several applications ranging from aerial photography to surveillance. In the recent years, the advancements in electronics has made cheaper and robust quadrotors available in the market. 

For a quadrotor to navigate autonomously in an unknown environment, it should map its surroundings, localize itself with respect to map, generate collision free paths from an initial position to a goal position, and accurately traverse that path. The last two requirements, known as \textit{path-planning} and \textit{control} are the main focuses of this study. 

The path-planning problem has been a topic of interest in various fields like robotics, artificial intelligence, and control theory. In robotics, path-planning was initially confined to simple geometric problems like the \textit{piano mover's problem}. Currently, it covers other complications such as non-holonomy, uncertainties, and dynamics. Over these years, several methods have been developed with each having their suitable applications. The three main types of approaches to this problem: (i) grid-based, (ii) potential-field based, and (iii) sampling-based. 

The path planning of a quadrotor is a high dimensional \textit{kinodynamic} problem. Sampling-based approaches have been demonstrated to be the best at dealing with them. Even though simple grid-based search are \textit{complete}, and can generate optimum paths, they fail to compute the solution in real time. Whereas, potential-field based approaches, though fast, are plagued by local minima. 

One crucial weakness of a quadrotor is its low flight time that is caused due to battery and size constraints. This would make it unsuitable for time-critical missions. However, this problem could be bypassed by maintaining a fleet of drones that share the same airspace to execute a mission. Some of the applications of such systems include autonomous warehouses and drone delivery. 

The multi-agent path planning problem is considerably more complex than the single-agent problem as it should consider robot-robot collisions during planning. This is an active field of research, and in recent years, several methods have been developed to compute sub-optimal paths in real-time. Most of these works rely on the fact that the interaction between agents are rare, and hence feasible paths could be found by searching a subset of the entire state-space. 

Once the path is generated, the robots should be able to track it accurately. Practically all the major control techniques, like linear, non-linear, adaptive, and predictive controllers have been applied to this end. Among these methods, Model Predictive Control (MPC) has generated impressive results by seamlessly incorporating the kinodynamic constraints as well as the dynamic obstacles.

The objective of this bibliography report is to present the state-of-the-art methods for path-planning with the main focus on the multi-agent quadrotor systems that has been implemented in literature. This work would act as the foundation to the Master Thesis titled: \textit{Trajectory planning of multiple fleets of robots using Model Predictive Control}. The aim of the thesis is to implement a real-time path planning and collision avoidance for multiple fleets of quadrotors with different priorities. 


This report is organized as follows: Chapter~\ref{chap:path} is dedicated to the basics of the path-planning problem. Various planning methodologies available in literature, along with their pros and cons, are presented in this chapter. In Chapter~\ref{chap:planning_quadrotors}, the popular path-planning and control strategies for quadrotors are discussed. Chapter~\ref{chap:multi} shows the challenges faced while extending these algorithms to multi-agent systems. Finally, the objectives and work plan of the Master Thesis are presented in Chapter~\ref{chap:work}.